%% Opprettet: 2012/10/20 08:04:58

\documentclass[11pt,english,a4paper]{article}
\newcounter{qcounter}            % Brukes sammen med setcounter (nedenfor)
\usepackage[T1]{fontenc,url}     % Standard fontencoding og url-encoding
\usepackage[sc,osf]{mathpazo}    % Font anbefalt av Dag Langemyhr
\usepackage[utf8]{inputenc}      % UTF encoding
% \usepackage[pdftex]{graphicx}  % For bilde og PDF
\usepackage{babel,textcomp,varioref}
%% babel    = alt som har med språk å gjøre
%% textcomp = utvidet sett med forskjellige symboler
%% varioref = for å bruke \vref

%% Matematikk og logikk
% \usepackage[mathcal]{euscript} % For potensmengde P (\mathcal{P})
% \usepackage{amsmath,amssymb,,upgreek}
%% amsmath  = matematikksymboler
%% amssymb  = yttligere matematikksymboler
%% upgreek  = smudere lambda symbol (\uplambda)

%% Tabell stæsj:
% \usepackage{booktabs,enumitem,tabularx}
% \usepackage{ltxtable,longtable}
%% booktabs = For \midrule, \toprule og \bottomrule til i fungere i tabeller
%% enumitem = For at [label=HVASOMHELST] skal fungere i enumerate
%% tabularx = For smidigere tabeller (bruk tabularx)
%% ltxtable = For å få tabeller over flere sider (og fordelen med tabularx)
%%   (bruk \LTXtable{width}{file.tex} for å laste inn fil med longtable env.)
%% longtable= Sammen med ltxtable for å få tabell over flere sider

%\pagestyle{empty}      % Fjern kommentering om du ikke ønsker sidenr.

%\usepackage{hyperref}  % Brukes for å lage hypertekst av TOC mm.
%\hypersetup{           % Utseende til hypertekst
%    colorlinks=true,
%    pdfborder={0 0 0}, % Viss 'colorlinks=false' og du ikke ønsker ramme
%    citecolor=blue,
%    filecolor=blue,
%    linkcolor=blue,
%    urlcolor=red
%}

\tolerance = 5000       % LaTeX er normalt streng når det gjelder linje-
\hbadness = \tolerance  % brytingen. Vi vil være litt mildere, særlig fordi
\pretolerance = 2000    % norsk har så mange lange sammensatte ord.

\title{INF3110 - Mandatory assignment 2}
\author{Mikael Olausson (mikaello)}
\begin{document}

\setcounter{secnumdepth}{-1} % Fjerner nummer fra overskrift, men TOC fungerer

\maketitle{}

\section{Implementation}

\subsection{Symbols used in grid}
I've used five different symbols inside the grid:
\begin{itemize}
  % I've used a little workaround here, i'm sorry for that.
\item[] \textperiodcentered{} \hspace{2 mm} default, this means empty space
\item[] < \hspace{2 mm} means \verb|-x|
\item[] > \hspace{2 mm} means \verb|x|
\item[] $\vee$ \hspace{2 mm} means \verb|-y|
\item[] $\wedge$ \hspace{2 mm} means \verb|y|
\end{itemize}
When pen is down, all steps are recorded with the correct direction
(depending on which way the robot was moving while recording).

\subsubsection{Example of code and grid}

\begin{verbatim}
down
start(2,3, x)
forward(3)
left(4)
backward(1)
right(2)
stop
\end{verbatim}

\begin{verbatim}
. . . . ^ . . . .
. . < < v . . . .
. . . . ^ . . . .
. . . . ^ . . . .
. > > > > . . . .
. . . . . . . . .
. . . . . . . . .
\end{verbatim}

\subsection{Coordinates }
I start coordinates from 1 and up to the specified size. This means
that if you specify a grid: \verb=size(5,5)=, coordinate $x=1,y=5$ is
valid, but not $x=0,y=4$.

\subsection{Pen-up/Pen-down}
When command \verb=down= is used (\emph{Pen-Down}), all the steps the
robot takes is stored in the first parameter of <<state>>, this
variable is updated with these steps.

\subsection{IfThenElse}
IfThenElse works without a else-statement, just pass a empty list.

\subsection{Attempts to move robot outside the grid}
When someone tries to move the robot outside the grid, a error-message
is written and an exception is raised (which makes the program quit
because it isn't handled).

\subsection{<<Pretty print>>}
I have made a function for pretty printing, this is called with all
the statements before execution of statements.

\section{TestPrograms}

I've added test-programs for the example given above, <<Testing code
1>> and <<Testing code 4>> (enabled as default)

\section{How to run the program }
You run the program by writing this in the terminal:
\begin{verbatim}
$ sml oblig2.sml
\end{verbatim}

\section{Get the code from GitHub}
This program is out on GitHub, you can get it with this command

\begin{verbatim}
$ git clone
   https://github.com/mikaello/inf3110_oblig2.git
\end{verbatim}

\end{document}
